\chapter{Inicialização}

Quando se utiliza o termo inicialização, dois significados aparecem: Inicialização da interface gráfica e inicialização da simulação propriamente dita. Este capítulo descreverá ambos, explicitando sobre o arquivos executáveis existentes no projeto e a maneira como são chamados. Inicialmente, será mostrado como é inicializado a interface gráfica e, posteriormente, como é inicializado  Gazebo.

\section{Inicializando interface gráfica}

Para inicializar a interface gráfica, utiliza-se o comando via terminal de link simbólico criado durante a instalação do sistema. \\

\noindent A partir do comando:

\begin{bashcode}
	provant_gui
\end{bashcode}

\noindent será executado o arquivo executável \textbf{GUI} localizado na pasta \$TILT\_PROJECT/source/build.sh Este arquivo é obtido após a execução do script de compilação \textit{install.sh} descrito no capítulo anterior.

\section{Inicializando simulação}

Após selecionado o cenário com seu respectivo VANT para simulação, é possível inicializar a simulação propriamente dita através do botão \textbf{Start Gazebo} localizado na tela inicial da interface gráfica como ilustrada na figura x. 

Este botão executa uma configuração de um comando no terminal de acordo com o tipo de simulação desejada, depende da seleção (ou não) pelo usuário da opção de simulação via hardware-in-the-loop. Caso queira executar uma simulação ``normal'', será executado  o comando \textbf{xterm e- roslaunch Database gazebo.lauch \&}. Já caso queira realizar uma simulação via Hardware-in-the-loop, será executado o comando \textbf{xterm e- roslaunch Database hil.lauch \&}. 

O comando \textbf{xterm e-} equivale a abrir um outro terminal no emulador de terminais xterm e executar o comando passado como paramêtro. Já o comando roslaunch é uma ferramenta utilizada para executar facilmente vários nós do ROS de acordo com a configuração prévia do arquivo launch passado como parâmetro. 

\subsection{Arquivos launch}

Arquivos launch são arquivos XML desenvolvidos com o intuito de realizar seleção e configuração de 

O primeiro executa o simulador gazebo e o nó controlador, já o segundo executará apenas o simulador gazebo. Na próxima seção será descrito cada um dos arquivos \textit{launch} e dos arquivos executáveis envolvidos na simulação 

\subsubsection{Localização no projeto}

