\chapter{Instalação}

O simulador ProVANT é um ambiente de simulação desenvolvido com o objetivo de validar e avaliar o desempenho de estratégias de controle. A versão do simulador a qual esse manual se refere foi desenvolvida sobre as plataformas Gazebo 7 e o \textit{framework} de desenvolvimento de aplicações robóticas \textit{Robot Operating System} (ROS), na versão Kinect. Para utilizá-la é necessário um computador com \textbf{sistema operacional Ubuntu 16.04}

O ROS fornece uma interface de programação para aplicações em robótica e dispõe de repositórios com vários módulos de software. Já o Gazebo é um software de simulação 3D e de licença gratuita, atualmente sob a responsabilidade da Open Source Robotics Foundation (OSRF). Ele é capaz de simular o comportamento dinâmico de corpos rígidos articulados, e inclui funcionalidades como detecção de colisão e visualização gráfica.

Este capítulo apresenta os passos necessários para instalação do simulador ProVANT. Inicialmente, são detalhados os processos de instalação de plataformas utilizadas pelo simulador, como a biblioteca QT versão 5, os pacotes de software Git e ROS. Por fim, é apresentado o processo de instalação do ambiente de simulação ProVANT. 

\section{Procedimentos para instalação do simulador}

Os procedimentos especificados a seguir assumem como premissa que o computador, o qual será instalado o simulador, está rodando uma versão limpa/recém-instalada do Ubuntu versão 16.04. %Assegure-se que o computador esteja formatado antes de prosseguir nesta seção.

\subsection{Instalando Git}

O código fonte do simulador ProVANT está hospedado no Github. Para o acesso a esses arquivos é necessário primeiro que o usuário tenha instalado no computador o pacote Git. Caso não tenha, para instalá-lo, abra um novo terminal e execute os seguintes comandos:

%O código fonte do ambiente de simulação ProVANT está hospedado no Github. Os procedimentos para download do código fonte, conforme explicitado posteriormente nesta seção assume que o usuário tenha instalado no computador o pacote Git. 

\begin{bashcode}
$ sudo apt-get update
$ sudo apt-get install git
\end{bashcode}

\subsection{Instalando e configurando o ROS}

O ROS fornece uma interface de programação para aplicações de robótica e vários módulos de software, entre eles o simulador Gazebo na versão 7, utilizado pelo simulador ProVANT. Esta subseção detalha as instruções necessárias para a instalação da distribuição ROS Kinetic Kame.  A seguir estão detalhados os passos para a instalação do ROS\footnote{para mais detalhes sobre o processo de instalação, assim como problemas na instalação pode-se acessar a homepage do ROS, \url{wiki.ros.org}}.\\
\bigskip
\bigskip
\bigskip

\textbf{Configure os repositórios do Ubuntu} \\

Antes de iniciar a instalação é necessário configurar os repositórios do Ubuntu para níveis de permissão "restricted", "universe", e "multiverse.". Atenção: normalmente, após a instalação do Ubuntu 16.04, estas opções já se encontram configuradas. Caso não estejam, para obter instruções sobre como fazer isso siga o passo a passo apresentado em:
\begin{center}
\url{https://help.ubuntu.com/community/Repositories/Ubuntu}.
\end{center}

\textbf{Configure o sources.list} \\

Também é necessário configurar o computador para aceitar o software de packages.ros.org. Para isso, abra um novo terminal e digite o seguinte comando:

\begin{bashcode}
$ sudo sh -c 'echo "deb http://packages.ros.org/ros/ubuntu $(lsb_release -sc) main"
 >/etc/apt/sources.list.d/ros-latest.list'
\end{bashcode}

\textbf{Informe as chaves de criptografia para acesso aos repositórios do ROS} \\

Isso pode ser realizado executando o seguinte comando:

\begin{bashcode}
$ sudo apt-key adv --keyserver hkp://ha.pool.sks-keyservers.net:80 --recv-key 
421C365BD9FF1F717815A3895523BAEEB01FA116
\end{bashcode}

\textbf{Instalação} \\

Antes de iniciar o processo de instalação do ROS, é necessário atualizar o índice do pacote Debian, para isso execute:

\begin{bashcode}
$ sudo apt-get update
\end{bashcode}

Tendo atualizado os pacotes Debian, pode-se fazer download dos arquivos binários do ROS

\begin{bashcode}
$ sudo apt-get install ros-kinetic-desktop-full
\end{bashcode}

\textbf{Inicialize o Rosdep} \\

Antes de poder usar o ROS, é necessário inicializar o sistema Rosdep.  O sistema Rosdep permite instalar as dependências do sistema, para o código fonte que deseja-se compilar. Ele é necessário para executar alguns dos componentes principais no ROS.

\begin{bashcode}
$ sudo rosdep init
$ rosdep update
\end{bashcode}

\textbf{Configuração de variáveis de ambiente} \\

Para que as variáveis de ambiente do ROS sejam automaticamente adicionadas sempre que um novo terminal é iniciado, para isso execute os seguintes comandos:

\begin{bashcode}
$ echo "source /opt/ros/kinetic/setup.bash" >> $HOME/.bashrc
$ source $HOME/.bashrc
\end{bashcode}


\textbf{Crie um espaço de trabalho ROS}\\

Por fim, é necessário criar um espaço de trabalho ROS, para isso execute a seguinte sequência de comandos no terminal:

\begin{bashcode}
$ mkdir -p $HOME/catkin_ws/src
$ cd $HOME/catkin_ws/
$ catkin_make
$ source $HOME/catkin_ws/devel/setup.bash
$ echo "source $HOME/catkin_ws/devel/setup.bash" >> $HOME/.bashrc
\end{bashcode}

\subsection{Instalando a biblioteca QT}

%Para a instalação apropriada da interface gráfica de configuração do ambiente de simulação ProVANT é necessário a instalação da \textit{Integrated Development Environment} (IDE) Qt Creator 5. Para isso, acesse a url a seguir e faça download do pacote de instalação:
%\begin{center}
%\url{https://www.qt.io/download-open-source/?hsCtaTracking=f977210e-de67-475f-a32b-65cec207fd03%7Cd62710cd-e1db-46aa-8d4d-2f1c1ffdacea} 
%\end{center}

%Atenção: para realizar a instalação do pacote (com extensão ``.run''), é necessário fornecer permissão de execução ao arquivo de instalação. Assumindo que o arquivo se encontra na pasta ``\$HOME/Downloads'', com nome ``qt-unified-linux-x64-3.0.0-online.run'', este passo pode ser realizado a partir dos comandos: 

%\begin{bashcode}
%$ cd $HOME/Downloads
%$ sudo chmod +x qt-unified-linux-x64-3.0.0-online.run
%$ ./qt-unified-linux-x64-3.0.0-online.run
%\end{bashcode}

Para a instalação apropriada da interface gráfica de configuração do ambiente de simulação ProVANT, é necessária a instalação da \textit{Integrated Development Environment} (IDE) Qt Creator 5. Para isso, execute os seguintes comandos no terminal: 	

\small
\begin{bashcode}
$ cd $HOME/Downloads
$ wget http://download.qt.io/official_releases/online_installers/qt-unified-linux-x64-online.run
$ sudo chmod +x qt-unified-linux-x64-online.run
$ ./qt-unified-linux-x64-online.run
\end{bashcode}
\normalsize


\subsection{Download e instalação do ambiente de simulação}

O código fonte do ambiente de simulação ProVANT, está localizado no repositório: 

\begin{center}
\url{https://github.com/Guiraffo/ProVANT-Simulator}
\end{center}

\noindent Este repositório possui acesso particular, então, antes de prosseguir como processo de instalação, solicite acesso ao Prof. Guilherme Vianna Raffo\footnote{E-mail para contato: raffo@ufmg.br}.

Com o acesso liberado, para fazer o download da versão atual do ambiente de simulação ProVANT, basta digitar no terminal os seguintes comandos:

\begin{bashcode}
$ cd $HOME/catkin_ws/src
$ git clone https://github.com/Guiraffo/ProVANT-Simulator.git
\end{bashcode}

Tendo realizado o download pode-se realizar a instalação e configuração do ambiente de simulação através da execução dos seguintes comandos:

\begin{bashcode}
$ cd $HOME/catkin_ws/src/ProVANT-Simulator
$ sudo chmod +x install.sh
$ ./install.sh
\end{bashcode}


Caso todos os procedimentos descritos anteriormente tenham sido realizados com sucesso, o usuário estará pronto para prosseguir para o próximo capítulo, referente ao fluxo de utilização do ambiente de simulação ProVANT.