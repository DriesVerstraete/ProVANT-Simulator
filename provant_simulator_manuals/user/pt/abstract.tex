\chapter*{Resumo}

O objetivo deste manual é descrever os procedimentos necessários para a utilização do simulador ProVANT. Nele é apresentado desde o processo de instalação até a realização de testes de estratégias de controle via simulação. O Texto está organizado como a seguir:

\begin{enumerate}
	\item O primeiro capítulo apresenta o contexto no qual esse ambiente de simulação está inserido e também uma breve introdução à VANTs.
	\item O segundo capítulo apresenta os passos necessários para instalação do ambiente de simulação ProVANT;
	\item O terceiro capítulo descreve o fluxo de utilização do ambiente de simulação, detalhando todas as funcionalidades da interface gráfica, tais como a escolha do cenário, modelo, estrategia de controle e aeronave;
	\item O quarto capítulo é o mais importante para o usuário. Ele descreve os procedimentos para a implementação de novas estratégias de controle no ambiente de simulação. 
	%Este é o capítulo mais importantes deste manual;
	\item O Apêndice A explica a estrutura de arquivos por trás dos modelos dos VANTs utilizados. Nesse capítulo são apresentadas as principais informações sobre modelagem cinemática e a importação de arquivos do software CAD para a plataforma Gazebo; 
	\item O Apêndice B descreve os plugins utilizados no ambiente de simulação. 
	%Os plugins são bibliotecas dinâmicas carregadas em tempo de execução, utilizadas para a implementação de sensores e atuadores no ambiente de simulação.; 
	\item Os Apêndices C e D descrevem os arquivos CMakelists.txt e package.xml. 
\end{enumerate}